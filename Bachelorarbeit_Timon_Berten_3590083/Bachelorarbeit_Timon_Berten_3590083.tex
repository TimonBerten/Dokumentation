% Vorlage für eine IRS-Bachelor/Masterarbeit
% Hinweis: Die Vorlage ist als Hilfestellung für die Erstellung einer Abschlussarbeit gedacht und stellt keinenfalls eine garantierte Einhaltung der Institutsrichtlinien dar. Viel Erfolg bei der Arbeit!
% Autor: Jan-Steffen Fischer nach einer Vorlage von Richard Kastelik

% Dokumentenklasse, Schriftgröße (11-12 empfohlen), Sprache, Zweiseitiges Dokument 
\documentclass[11pt,ngerman,open=right,twoside=true]{scrbook}

%Input Codierung
\usepackage[utf8]{inputenc}
% Seitenlayout A4
\usepackage[a4paper]{geometry}
% Seitenränder (Korrekturrand 2,5 lt. Richtlinien)
\geometry{verbose,tmargin=2.4cm,bmargin=3.2cm,lmargin=2.6cm,rmargin=2.2cm}
% Numerierungstiefe einstellen 
\setcounter{secnumdepth}{3}
%Absatzlänge einstellen
\setlength{\parskip}{\bigskipamount}
\setlength{\parindent}{0pt}
%Sprachpaket
\usepackage{babel}

%Zitation
\usepackage[babel, german=quotes]{csquotes}
\usepackage[backend=biber, style=numeric, firstinits=true, sorting=none]{biblatex}
% Wichtig: die Datei muss mit Biber kompliliert werden. Erst die tex datei kompilieren, danach unter Tools --> Befehle --> Biber auswählen und anschließend nochmals die tex kompilieren. 

% Nachname in Kapitlächen
%\renewcommand*{\mkbibnamelast}[1]{\textsc{#1}}
% erst Nachname, dann Vorname
\DeclareNameAlias{default}{last-first}
% Trenner zwischen den Namen ein Semikolon
\renewcommand*{\multinamedelim}{\addsemicolon\space}
\renewcommand*{\finalnamedelim}{\addsemicolon\space}
% Trenner zwischen den Namen ein Semikolon
\renewcommand*{\multinamedelim}{\space}
% Doppelpunkt nach dem letzten Namen
\renewcommand*{\labelnamepunct}{\addcolon\space}
% Ausgabe der Annotation
%\renewcommand*{\annotationfont}{\small\itshape}
\renewbibmacro*{finentry:annotation}{%
  \iffieldundef{annotation}
    {\finentry}%
    {\setunit{\addperiod\space}% statt: \addperiod\par
     \printfield{annotation}}%
}

% Literaturdatei
\addbibresource{literatur.bib}

%PDF-Dokumente Verweise
\usepackage[unicode=true,
 bookmarks=true,bookmarksnumbered=false,bookmarksopen=false,
 breaklinks=false,pdfborder={0 0 1},backref=false,colorlinks=false,urlcolor=blue]
 {hyperref}
\hypersetup{pdftitle={Titel},
 pdfauthor={Student}}

%Tabelle wird zu Tab. und Abbildung zu Abb.
\addto\extrasgerman{\renewcommand{\figurename}{Abb.}}
\addto\extrasgerman{\renewcommand{\tablename}{Tab.}}

% Bilder
\usepackage{graphicx}
\usepackage[rflt]{floatflt}
\usepackage{epsfig,wrapfig}
\usepackage{float}
\usepackage{subfig}
\usepackage[section]{placeins}
\usepackage{rotating}
%\usepackage{subcaption}

% Mathematische Symbole
\usepackage{amsmath,amssymb}
\usepackage[official]{eurosym}
\usepackage{romannum}

% Tabellen
\usepackage{longtable,lscape}
\usepackage{multirow}
\usepackage{tabularx}
%horizontale Tabellen
\usepackage{lscape}
\newcommand\tabrotate[1]{\rotatebox{90}{#1\hspace{\tabcolsep}}}
\newcommand\verschiebung[1][-.75\normalbaselineskip]{\hspace{#1}}

% Listenerscheinung
\setlength{\itemsep}{0ex}
\setlength{\parsep}{0ex}
\setlength{\parskip}{2mm}

% Einbindung Aufgabenstellung als PDF ermöglichen
\usepackage{pdfpages}

%%% Entwurfsstatus
% Wasserzeichen
%\usepackage[firstpage]{draftwatermark}
%\SetWatermarkText{Entwurf vom: \today}

% To-Do Notes
\usepackage{todonotes}

%Auskommentierung
\usepackage{comment}

%%%

% weist dem Zeichen @ Kategorie-Code 11 zu, so dass es in Namen von Befehlssequenzen verwendet werden kann
\makeatletter
% Vermeidet neue/alte Absätze an Seitenanfängen / -enden
\widowpenalty=3000
\clubpenalty=3000
\brokenpenalty=100000

%Silbentrennung bearbeiten
\usepackage{hyphenat}

%Manipulation von Kopf- bzw. Fußzeile
\usepackage{fancyhdr}
\pagestyle{fancy}
%Definiert die Dicke der Kopf/Fußzeile
\renewcommand{\headrulewidth}{0.5pt}
\renewcommand{\footrulewidth}{0.5pt}
%Einstellung wo Kapitel/Sektion angegeben werden
\renewcommand{\chaptermark}[1]{\markboth{\thechapter{} #1}{}}
\renewcommand{\sectionmark}[1]{\markright{\thesection{} #1}}
%Clears the header and footer,
\fancyhf{}
% Seitenzahlen Links/Rechts 
\fancyhead[EL]{\textsf{\textbf{\leftmark}}}
\fancyhead[OR]{\textsf{\textbf{\rightmark}}}
\fancyfoot[EL,OR]{\textsf{\textbf{\thepage}}}
%\fancyhead[EL,OR]{\textbf{\thepage}}


% Einstellungen für plain Seiten
\fancypagestyle{plain}
{
\renewcommand{\headrulewidth}{0.5pt}
\renewcommand{\footrulewidth}{0.5pt}

\fancyhf{}

\fancyhead[EL]{\textsf{\textbf{\leftmark}}}
\fancyhead[OR]{\textsf{\textbf{\rightmark}}}
\fancyfoot[EL,OR]{\textsf{\textbf{\thepage}}}
%\fancyhead[EL,OR]{\textbf{\thepage}}
}
%Bilder werden in der Section ausgegeben
\usepackage[section]{placeins}
%weist @ der Kategorie-Code 12 zu
\makeatother

%%%%%%%%%%% Dokument

\begin{document}
% Titelseite
\thispagestyle{empty}

% Abstände müssen ggf angepasst werden, damit Titel in das Fenster passt
\vspace*{23mm}
\hspace{30mm}%
\begin{minipage}[c][67mm]{117mm}%
% Titel in Originalsprache
\begin{center}
	\textbf{\Large{}Entwurf des Entfaltmechanismus für das Kamera-System der DETECT Nutzlast für die SOURCE-2 Mission}{\Large\par}
	\par\end{center}
% Titel in englischer/deutscher Sprache
\begin{center}
	\textbf{\textit{\Large{Design of the deployment mechanism for the camera system of the DETECT payload for the SOURCE-2 mission}}}{\Large\par}
	\par\end{center}
\begin{center}
Bachelorarbeit von\\
Timon Berten
\par\end{center}
\begin{center}
IRS-Nummer
\par\end{center}%
\end{minipage}

\vspace*{3.5cm}

\hspace{30mm}%
\begin{minipage}[c]{117mm}%
\begin{center}
\textbf{\small{}Hochschullehrerin:}{\small{}}\\
{\small{}Frau Prof. Dr.-Ing. Sabine Klinkner}{\small\par}
\par\end{center}
\begin{center}
\textbf{\small{}Betreuer/in:}{\small{}}\\
{\small{}(Unternehmen: Name des externen Betreuers/der externen Betreuerin \& akademischer Titel)}\\
{\small{}Institut für Raumfahrtsysteme: Marlin Kanzow M.Sc.}\\
{\small{}\vspace*{3.5cm}}{\small\par}
\par\end{center}
\begin{center}
(Unternehmen, Ort) \\
Institut für Raumfahrtsysteme\\
Universität Stuttgart
\par\end{center}
\begin{center}
April 2025
\par\end{center}%
\end{minipage}
\newpage{}


\cleardoublepage
%To-Do: Aufgabenstellung einfügen
%\includepdf[pages=1]{Aufgabenstellung}
\cleardoublepage
%To-Do: Erklärungen einfügen
%\includepdf[pages=1]{Bestaetigung}


\newpage
%\listoftodos
\pagenumbering{Roman} \setcounter{page}{5}
%Inhaltsverzeichnis
\tableofcontents{}

%Abbildungsverzeichnis (optional)
%\listoffigures

%Tabellenverzeichnis (optional)
%\listoftables
%\newpage
%\thispagestyle{empty}
%\quad

% Nomenklatur
\markboth{Nomenklatur}{Nomenklatur}
%\addcontentsline{toc}{chapter}{Nomenklatur}
\newpage
\chapter*{Nomenklatur}

Dies ist ein Beispiel einer Nomenklatur, näheres siehe Kap. \ref{chap:aufbau}.

\section*{Formelzeichen}
\begin{tabular}{lll}
  $c$            & $\mathrm{m/s}$ & Geschwindigkeit der Triebwerksabgase\\
  \vspace{1mm}
  $m$            & kg & Masse \\
  \vspace{1mm}
  $v$            & $\mathrm{m/s}$ & Geschwindigkeit des Raumschiffs\\
  \vspace{1mm}
\end{tabular}

\begin{tabular}{lll}
	$\mu$            & - & Nutzlastverhältnis \\
	\vspace{1mm}
	$\sigma$            & - & Strukturmassenanteil \\
	\vspace{1mm}
\end{tabular}

\section*{Indizes}
\begin{tabular}{lll}
	e  &  & effektiv \\
	\vspace{1mm}
	T  &  & Treibstoff \\
	\vspace{1mm}
	0  &  & Gesamt \\
	\vspace{1mm}
\end{tabular}

\section*{Konstanten}
\begin{tabular}{lll}
	$\gamma$  & 6,674$\cdot10^{-11}\,\mathrm{m^3/kgs^2}$  & Gravitationskonstante \\
	\vspace{1mm}
	$\mathrm{M_{\textrm{E}}}$  & 5,974$\cdot10^{24}\,$kg & Masse der Erde \\
	\vspace{1mm}
	$\mathrm{R_{\textrm{E}}}$  & 6,378$\cdot10^6\,$m & Radius der Erde \\
	\vspace{1mm}
\end{tabular}

\section*{Abkürzungen}
\begin{tabular}{lll}
\vspace{1mm}
  Abb.            &            & Abbildung \\
  \vspace{1mm}  
  CoG            &            & Schwerpunkt des Gesamtsystems \\
  \vspace{1mm}
  CoP            &           & Angriffspunkt der aerodynamischen Kräfte\\
  \vspace{1mm}
  REACH            &            & Registration, Evaluation, Authorisation and Restriction of Chemicals \\
  \vspace{1mm} 
  Tab.            &            & Tabelle \\
  \vspace{1mm}    
\end{tabular}


\cleardoublepage
\pagenumbering{arabic}

%Einführung in die Source 2 Mission und Mechanismen in der Raumfahrt
\newpage
\chapter{Einleitung}
\section{Source 2 Mission}
\subsection{XX}
XX


\section{Mechanismen in der Raumfahrt}
\subsection{XX}
XX




%Grundlagen zu CubeSats und den einzelnen Bauteilen für die Mechanismen
\chapter{Grundlagen}
XX



\section{CubeSat und Exolaunch}

XX

\section{Grundlagen für Bauteile der Booms und ECSS Standart}
XX




% Formalien rund um die Abschlussarbeit
\newpage
\chapter{Mechanismus Requirements}
XX


\section{Strukturelle}

XX

\section{Payload}
XX

%Vorstellung aller DB Konzepte und anschließendes TradeOff
\newpage
\chapter{Mechanismen Konzepte und Trade Off}
XX


\section{Konzept 1}
\section{Konzept 2}
\section{Konzept 3}
\section{Konzept 4}
\section{Konzept 5}


XX

\section{Trade Off}
XX

% Tests und Simulation in Matlab
\newpage
\chapter{Nachweis des Mechanismus}
XX


\section{xx}

XX

\section{xx}
XX

%Darstellung des Gesamtentwurf / Funktionsweise und Details
\newpage
\chapter{Detaillierter Gesamtentwurf}
XX


\section{Aufbau}

XX

\section{Funktionsweise}
XX

% Zusammenfassung und Ausblick
\newpage
\chapter{Zusammenfassung und Ausblick}
XX


% Literaturverzeichnis
\printbibliography

\cleardoublepage
\appendix
\markboth{}{}
% Fortführung der lateinischen Numerierung
\pagenumbering{Roman}
\setcounter{page}{9}

% Anhang
\newpage
\chapter*{Anhang}

\chapter{Vorlage}
Der Latex-Code dieser Vorlage kann unter folgendem Link heruntergeladen werden:\\
\url{https://ncext.irs.uni-stuttgart.de/index.php/s/PBdKPHtcsWgxZ7w}\\
Passwort: RaumfahrtMachtSpass.2024

\chapter{Nützliche Programme}

\section{Latex-Editor}
\label{sec:latex_editor}
\begin{itemize}
	\item Texstudio: \url{https://www.texstudio.org/} oder Texmaker: \url{https://www.xm1math.net/texmaker/} 
	\item Miktex: \url{https://miktex.org/}, muss dazu installiert werden (Paketbibliothek)
\end{itemize}
\section{Datenauswertung \& Diagramme}
\label{sec:datenauswertung_diagramme}
\begin{itemize}
	\item Python (Pandas) mit Spyder in der Anaconda-Umgebung: \url{https://www.anaconda.com/}
	\item Matplotlib-Paket für Python für Diagramme
	\item Matlab 
\end{itemize}
\section{Abbildungen}
\label{sec:abbildungen}
\begin{itemize}
	\item Inkscape (Vektorgrafiken): \url{https://inkscape.org/de/}
	\item GIMP (Rastergrafiken): \url{https://www.gimp.org/}
\end{itemize}
\section{Literaturverwaltung}
\label{sec:literaturverwaltung}
\begin{itemize}
	\item Zotero: \url{https://zotero.org/}
	\item Citavi: \\ \url{https://www.ub.uni-stuttgart.de/lernen-arbeiten/literaturverwaltung-mit-citavi/}
\end{itemize}

\end{document}
