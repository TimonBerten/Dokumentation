\newpage
\chapter*{Nomenklatur}

Dies ist ein Beispiel einer Nomenklatur, näheres siehe Kap. \ref{chap:aufbau}.

\section*{Formelzeichen}
\begin{tabular}{lll}
  $c$            & $\mathrm{m/s}$ & Geschwindigkeit der Triebwerksabgase\\
  \vspace{1mm}
  $m$            & kg & Masse \\
  \vspace{1mm}
  $v$            & $\mathrm{m/s}$ & Geschwindigkeit des Raumschiffs\\
  \vspace{1mm}
\end{tabular}

\begin{tabular}{lll}
	$\mu$            & - & Nutzlastverhältnis \\
	\vspace{1mm}
	$\sigma$            & - & Strukturmassenanteil \\
	\vspace{1mm}
\end{tabular}

\section*{Indizes}
\begin{tabular}{lll}
	e  &  & effektiv \\
	\vspace{1mm}
	T  &  & Treibstoff \\
	\vspace{1mm}
	0  &  & Gesamt \\
	\vspace{1mm}
\end{tabular}

\section*{Konstanten}
\begin{tabular}{lll}
	$\gamma$  & 6,674$\cdot10^{-11}\,\mathrm{m^3/kgs^2}$  & Gravitationskonstante \\
	\vspace{1mm}
	$\mathrm{M_{\textrm{E}}}$  & 5,974$\cdot10^{24}\,$kg & Masse der Erde \\
	\vspace{1mm}
	$\mathrm{R_{\textrm{E}}}$  & 6,378$\cdot10^6\,$m & Radius der Erde \\
	\vspace{1mm}
\end{tabular}

\section*{Abkürzungen}
\begin{tabular}{lll}
\vspace{1mm}
  Abb.            &            & Abbildung \\
  \vspace{1mm}  
  CoG            &            & Schwerpunkt des Gesamtsystems \\
  \vspace{1mm}
  CoP            &           & Angriffspunkt der aerodynamischen Kräfte\\
  \vspace{1mm}
  REACH            &            & Registration, Evaluation, Authorisation and Restriction of Chemicals \\
  \vspace{1mm} 
  Tab.            &            & Tabelle \\
  \vspace{1mm}    
\end{tabular}